\input{../../templates/assignment.tex}

\title{	
\normalfont \normalsize 
\textsc{Norwegian University of Science and Technology\\TDT4200 -- Parallel Computing} \\ [25pt]
\horrule{0.5pt} \\[0.4cm]
\huge Problem Set 3:\\ Debugging \& Optimization\\
\horrule{2pt} \\[0.5cm]
}

\author{Per Magnus Veierland\\permve@stud.ntnu.no}

\setlist[enumerate,1]{label=\emph{\alph*})}
\setlist[enumerate,2]{label=\roman*)}
\setlist[enumerate,3]{label=\arabic*)}

\date{\normalsize\today}

\begin{document}
\maketitle

\section*{Part 2: Code}

\subsection*{Problem 1: Debugging}

\begin{enumerate}

\item The program \texttt{test01.c} has the following bugs:

\begin{enumerate}[label=Bug \#\arabic*)]
\item
The dynamic memory allocated at \texttt{test01.c:6} is never freed.
\item
The dynamic memory allocated at \texttt{test01.c:6} is never used.
\end{enumerate}

\item The program \texttt{test02.c} has the following problems:

\begin{enumerate}[label=Problem \#\arabic*)]
\item
The dynamic memory allocated at \texttt{test02.c:6} is never freed.
\item
The array accesses seem a bit illogical, although it is hard to know the exact indented behavior. Currently array indexes 0-31 are initialized with the values 0-31, and the contents of array indexes 64-95 is copied from array indexes 0-31. This leaves array indexes 32-63 and 96-127 uninitialized.

The more logical behavior would likely be for each loop to iterate from $i=0$ to $i<64$ which would result in array indexes 0-63 and 64-127 both to be set to the values with in the range 0-63.
\end{enumerate}

\end{enumerate}

\end{document}

