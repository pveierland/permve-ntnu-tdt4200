\input{../../templates/assignment.tex}

\title{	
\normalfont \normalsize 
\textsc{Norwegian University of Science and Technology\\TDT4200 -- Parallel Computing} \\ [25pt]
\horrule{0.5pt} \\[0.4cm]
\huge Problem Set 7:\\ Parallelization\\
\horrule{2pt} \\[0.5cm]
}

\author{Per Magnus Veierland\\permve@stud.ntnu.no}

\setlist[enumerate,1]{label=\emph{\alph*})}
\setlist[enumerate,2]{label=\roman*)}
\setlist[enumerate,3]{label=\arabic*)}

\date{\normalsize\today}

\begin{document}

\maketitle

To determine whether it is necessary to apply parallelism within the \texttt{performNewIdeaIteration} function, or if it is possible to handle the parallelism outside this function, initial timing measurements were performed. The \texttt{main} function has four sections of code calling \texttt{performNewIdeaIteration}. Measuring the time it takes to execute each of these sections using \texttt{pclock_gettime(CLOCK\_MONOTONIC)} and averaging the timings from 5 runs on an \textsc{ITS015} machine after warming the system caches yields the following values:

\begin{displaymath}
86.199800~\text{ms} 80.838335~\text{ms} 82.280961~\text{ms} 81.698863~\text{ms}
\end{displaymath}

The averaged measurements for each code section has a standard deviation of 2~ms, which is about 2.5\% of their mean value, which shows that the sections take about the same time to execute.

\end{document}

