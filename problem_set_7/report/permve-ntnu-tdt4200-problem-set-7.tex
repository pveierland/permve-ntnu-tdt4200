\input{../../../permve-ntnu-latex/assignment.tex}

\title{
\normalfont \normalsize
\textsc{Norwegian University of Science and Technology\\TDT4200 -- Parallel Computing} \\ [25pt]
\horrule{0.5pt} \\[0.4cm]
\huge Problem Set 7:\\ Parallelization\\
\horrule{2pt} \\[0.5cm]
}

\author{Per Magnus Veierland\\permve@stud.ntnu.no}

\setlist[enumerate,1]{label=\emph{\alph*})}
\setlist[enumerate,2]{label=\roman*)}
\setlist[enumerate,3]{label=\arabic*)}

\date{\normalsize\today}

\begin{document}

\maketitle

\section*{MPI}

The parallel MPI implementation is specified to utilize exactly four MPI ranks. In the main function, four separate rounds of filtering is applied; one for each filter size. To determine whether the work performed by these four filters is equal, measuring code was added to the program through the use of \texttt{clock\_gettime(CLOCK\_MONOTONIC)} which returns monotonic time measurements with nanosecond resolution. Averaging timing measurements for the four filter sizes over five runs on an \textsc{ITS015} machine yielded the following values:\\[0.2cm]
Filter size 5: 86.2~ms\quad Filter size 7: 80.84~ms\quad Filter size 11: 82.28~ms\quad Filter size 17: 81.70~ms\\[0.2cm]
The averaged measurements for each code section has a standard deviation of 2~ms, which is about 2.5\% of their mean value, which shows that the sections take about the same time to execute.

The work performed by each rank is then to read the source \texttt{flower.ppm} image file and perform filtering for one filter size. Each rank other than the first will then asynchronously transfer its result to the its lower rank neighbor using a non-blocking synchronous send (\texttt{MPI\_Issend}). Each rank other than the last initiates a non-blocking receive (\texttt{MPI\_Irecv}) before computing the filtered image. In the end all receiving ranks waits for the receive to complete using \texttt{MPI\_Wait}, before processing the combined image from their own image and the image received from the rank above, writing the result to the output file.

\end{document}

